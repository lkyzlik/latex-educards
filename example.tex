%%%%%%%%%%%%%%%%%%%%%%%%%%%%%%%%%%%%%%%%%%%%%%%%%
% Card front
%%%%%%%%%%%%%%%%%%%%%%%%%%%%%%%%%%%%%%%%%%%%%%%%%
\begin{educardf}{RoyalBlue}{\faShapes}{Model}{Bloom's taxonomy}{How to set your teaching outcomes?}{bloom.png}

Bloom's Taxonomy is a hierarchical framework for classifying educational learning objectives, developed by Benjamin Bloom in 1956. It consists of six cognitive levels: Remember (recall facts), Understand (explain concepts), Apply (use information in new situations), Analyze (draw connections), Evaluate (justify positions), and Create (produce original work). Each level builds upon the previous one, moving from basic knowledge acquisition to complex critical thinking. Teachers use this framework to design curriculum.

\vspace{20pt}

% Footer
\rule{\textwidth}{0.4pt}

\textit{ \faGlobe \hspace{1pt} \href{wikipedia.org/wiki/Bloom\%27s_taxonomy}{Wikipedia: Bloom's Taxonomy}}

\end{educardf}

%%%%%%%%%%%%%%%%%%%%%%%%%%%%%%%%%%%%%%%%%%%%%%%%%
% Card back
%%%%%%%%%%%%%%%%%%%%%%%%%%%%%%%%%%%%%%%%%%%%%%%%%
\begin{educardb}{RoyalBlue}{\faShapes}{Model}{Bloom's taxonomy}

\begin{itemize}[leftmargin=4mm]

\item \textbf{Remember:} The ability to recall previously learned material, from facts to theories. Involves remembering terminology, conventions, sequences, and methodologies. \textit{Key actions:} define, list, recall, recognize, identify.

\item \textbf{Understand:} The ability to grasp material's meaning by translating, interpreting, and predicting consequences. Students demonstrate this by explaining and summarizing information. \textit{Key actions:} explain, interpret, describe, compare.

\item \textbf{Apply:} The ability to use learned material in new situations, including applying rules, methods, and theories in practical scenarios. \textit{Key actions:} solve, demonstrate, use, compute.

\item \textbf{Analyze:} The ability to break down material into components and understand relationships between parts. Involves examining evidence and identifying patterns. \textit{Key actions:} differentiate, analyze, examine, compare.

\item \textbf{Evaluate:} The ability to judge material's value based on specific criteria. Includes making judgments using internal evidence and external standards. \textit{Key actions:} assess, judge, critique, justify.

\item \textbf{Create:} The ability to combine elements to form new wholes. Involves producing original work, planning projects, and synthesizing knowledge. \textit{Key actions:} design, construct, develop, formulate.

\end{itemize}

Bloom's Taxonomy levels build progressively, with mastery of lower levels enabling success in higher ones. While hierarchical, the levels often operate simultaneously during complex learning tasks, enhancing overall cognitive development.

\end{educardb}